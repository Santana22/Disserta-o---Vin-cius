
% ---
% PACOTES BASICOS
% ---
\usepackage{lmodern}			% Usa a fonte Latin Modern			
\usepackage[T1]{fontenc}		% Selecao de codigos de fonte.
\usepackage[utf8]{inputenc}		% Codificacao do documento (conversão automática dos acentos)
\usepackage{lastpage}			% Usado pela Ficha catalográfica
\usepackage{indentfirst}		% Indenta o primeiro parágrafo de cada seção.
\usepackage{color}				% Controle das cores
\usepackage{graphicx}			% Inclusão de gráficos
\usepackage{microtype} 			% para melhorias de justificação


%------------------------------------------------
%\usepackage[brazil]{babel}
\usepackage{dsfont}
\usepackage{multirow}
\usepackage{multicol}
\usepackage{colortbl}
\usepackage{url}
%\usepackage{abntcite}
\usepackage{algorithm}
\usepackage{algorithmic}
%\usepackage{alg}
%\usepackage{hyperref}
\usepackage{caption} 
\usepackage{subcaption} % for subfigures

% para usar os comentários nos algoritmos
% conforme definido mais abaixo em:
% \renewcommand{\algorithmiccomment}[1]{\hfill\eqparbox{}{\# #1}}
\usepackage{eqparbox}

% exibe somente capitulo, seção e subseção
\setcounter{tocdepth}{2}

\newcommand{\abreviarEN}[2][abreviatura]{(\abrv[#1 -- #2]{#1}, do inglês \textit{#2})}

%------------------------------------------------



% ---
% PACOTES ADICIONAIS, USADOS APENAS NO ÂMBITO DO MODELO CANÔNICO DO abnteX2
% ---
\usepackage{lipsum}				% para geração de dummy text

% ---
% PACOTES DE CITAÇÕES
% ---
\usepackage[brazilian,hyperpageref]{backref}	 % Paginas com as citações na bibl
\usepackage[alf]{abntex2cite}	% Citações padrão ABNT

% ---
% PACOTES ADICIONADOS POR CEPHAS
% ---
\usepackage{float}
\usepackage{amssymb,amsmath}
\usepackage{pdfpages}
\usepackage{acronym}

% --- 
% CONFIGURAÇÕES DE PACOTES
% --- 

% Configurações do pacote backref
% Usado sem a opção hyperpageref de backref
\renewcommand{\backrefpagesname}{Citado na(s) página(s):~}
% Texto padrão antes do número das páginas
\renewcommand{\backref}{}
% Define os textos da citação
\renewcommand*{\backrefalt}[4]{
	\ifcase #1 %
		Nenhuma citação no texto.%
	\or
		Citado na página #2.%
	\else
		Citado #1 vezes nas páginas #2.%
	\fi}%
% ---

% utilizando o todo notes para deixar comentários
\setlength {\marginparwidth }{2cm}
\usepackage[colorinlistoftodos]{todonotes}
%\usepackage[disable,colorinlistoftodos]{todonotes}
\newcommand{\duvida}[1]{\todo[color=yellow, inline, textcolor=black]{\texttt{Duvida:} #1}}
\newcommand{\lembrete}[1]{\todo[color=lightgray, inline, textcolor=black]{\texttt{Lembrete:} #1}}
\newcommand{\parafazer}[1]{\todo[color=green, inline, textcolor=black]{\texttt{Para Fazer:} #1}}

% Macros para indicar modificações no texto
\usepackage[normalem]{ulem}
\newcommand{\removido}[1]{\textcolor{red}{\sout{#1}}}
\newcommand{\novo}[1]{\textcolor{blue}{#1}}

% Macros para remover as modificações de texto na versão final
% comentar acima e descomentar abaixo na versão final
% \newcommand{\removido}[1]{}
% \newcommand{\novo}[1]{#1}
% \renewcommand*{\sout}{}

% Para poder ter tabelas com colunas de largura auto-ajustável
\usepackage{tabularx}

% Redefinicao de instrucoes
\floatname{algorithm}{Algoritmo}
\renewcommand{\algorithmicrequire}{\textbf{Entrada:}}
\renewcommand{\algorithmicensure}{\textbf{Saída:}}
\renewcommand{\algorithmicend}{\textbf{fim}}
\renewcommand{\algorithmicif}{\textbf{se}}
\renewcommand{\algorithmicthen}{\textbf{então}}
\renewcommand{\algorithmicelse}{\textbf{senão}}
\renewcommand{\algorithmicfor}{\textbf{para}}
\renewcommand{\algorithmicforall}{\textbf{para todo}}
\renewcommand{\algorithmicdo}{\textbf{faça}}
\renewcommand{\algorithmicwhile}{\textbf{enquanto}}
\renewcommand{\algorithmicloop}{\textbf{loop}}
\renewcommand{\algorithmicrepeat}{\textbf{repetir}}
\renewcommand{\algorithmicuntil}{\textbf{até que}}
\renewcommand{\algorithmiccomment}[1]{\hfill\eqparbox{}{\# #1}}

% Definicao da lista de simbolos
% \simb[entrada na lista de simbolos]{simbolo}:
% Escreve o simbolo no texto e uma entrada na lista de simbolos.
% Se o parametro opcional e omitido, usa-se o parametro obrigatorio.
\newcommand{\simb}[2][]
{%
	\ifthenelse{\equal{#1}{}}
	{\addcontentsline{los}{simbolo}{#2}}
	{\addcontentsline{los}{simbolo}{#1}}#2
}
% Para aceitar comandos com @ (at) no nome
\makeatletter 
% \listadesimbolos: comando que imprime a lista de simbolos
\newcommand{\listadesimbolos}
{
	\pretextualchapter{Lista de símbolos}
	{\setlength{\parindent}{0cm}
	\@starttoc{los}}
}
% Como a entrada sera impressa
\newcommand\l@simbolo[2]{\par #1}
\makeatother

% Definicao da lista de abreviaturas e siglas
% \abrv[entrada na lista de simbolos]{abreviatura}:
% Escreve a sigla/abreviatura no texto e uma entrada na lista de abreviaturas e siglas.
% Se o parametro opcional e omitido, usa-se o parametro obrigatorio.
\newcommand{\abrv}[2][]
{%
	\ifthenelse{\equal{#1}{}}
	{\addcontentsline{loab}{abreviatura}{#2}}
	{\addcontentsline{loab}{abreviatura}{#1}}#2
}
% Para aceitar comandos com @ (at) no nome
\makeatletter 
% \listadeabreviaturas: comando que imprime a lista de abreviaturas e siglas
\newcommand{\listadeabreviaturas}
{
	\pretextualchapter{Lista de abreviaturas e siglas}
	{\setlength{\parindent}{0cm}
	\@starttoc{loab}}
}
% Como a entrada sera impressa
\newcommand\l@abreviatura[2]{\par #1}
\makeatother

% \listofalgorithms: comando que imprime a lista de algoritmos
\renewcommand{\listalgorithmname}{Lista de algoritmos}
