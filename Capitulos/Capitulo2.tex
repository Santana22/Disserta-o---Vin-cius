% Capítulo 2
%----------------------------------------------
\chapter{Referencial teórico}
\label{cap:capitulo2}
% label do capítulo para usar como referência no texto
%----------------------------------------------

Breve resumo do que será apresentado no capítulo.

 %----------------------------------------------
\section{Dicas para utilização do \LaTeX}
%----------------------------------------------

Parágrafo de introdução.

%----------------------------------------------
\subsection{Dicas gerais}
%----------------------------------------------

\textbf{Não} utilizar aspas duplas, exemplo "aspas".  Utilizar duas crases para abrir aspas (\`{}\`{}) e duas aspas simples (\textquotesingle\textquotesingle) para fechar. Resultado fica ``assim''. 

No \LaTeX, o til (\~{}) pode ser utilizado para significar um espaço em branco sem permitir a quebra de linha neste local. Ou seja, quando utilizado desta forma ``\verb|Tabela~\ref{tab:exemplo}|'', não vai haver quebra de linha entre a palavra ``Tabela'' e o seu número ``1'', por exemplo. Desta forma, é recordável a utilização em todas as referências cruzadas (tabelas, figuras, algoritmos, quadros, etc) (e.g., \verb|~\ref{}|) e também em citações (e.g., \verb|~\cite{}|).


%----------------------------------------------
\subsection{Referências a figuras, tabelas, seções}
\label{sec:usarLabels}
%----------------------------------------------

Utilizar a tag \verb|\label{}| para adicionar nomes aos elementos e depois usar a tag \verb|\ref{}| para referenciar. Exemplo referenciando a Seção~\ref{sec:usarLabels}. 

%----------------------------------------------
\subsection{ToDo Notes}
%----------------------------------------------

Utilizar as macros de notas. 

\verb|\parafazer{}|
\parafazer{Atividade/Ações/Escritas que ainda precisam ser realizadas.}

\verb|\duvida{}|
\duvida{Deixe sua duvida registrada aqui.}

\verb|\lembrete{}|
\lembrete{Deixar um comentário, uma observação.}

Não é necessário apagar ou comentar as notas manualmente para que elas não apareçam no PDF, basta adicionar ``disable'' na inserção do pacote.

\begin{verbatim}
Arquivo -> dissertacao.tex 
Linha ->   \usepackage[disable,colorinlistoftodos]{todonotes}.
\end{verbatim}

%---------------------------------------------- 
\subsection{Notas de rodapé}
%----------------------------------------------

Texto\footnote{https://www.google.com} é um mecanismo de automação utilizado
	    
%----------------------------------------------
\subsection{Figuras}
%----------------------------------------------

A Figura~\ref{fig:figuraTeste} exibe ...

O código de inserção das figuras deve \textbf{sempre} estar localizado abaixo da primeira referência a esta. O próprio \LaTeX vai posicionar a imagem. Utilizar a tag \verb|[!htb]|.

% exemplo de figura
\begin{figure}[!htb]
    \centering
    \includegraphics[width=\textwidth]{example-image-a}
    % pode tem ser definido o tamanho da figura 
    %width = largura
    %height = altura
    %\includegraphics[width=10cm,height=10cm]{Imagens/FiguraTeste.png}
    \caption{Legenda da figura} 
    % label é utilizado para referenciar a figura no texto
    % A Figura~\ref{fig:figuraTeste} exibe ...
    \label{fig:figuraTeste}
\end{figure}


Para referenciar a figura inteira, utilizar \verb|~\ref{fig:2imagens}|, ex. Figura~\ref{fig:2imagens}. Para referenciar somente A, utilizar \verb|~\ref{fig:labelFigA}|, ex. Figura~\ref{fig:labelFigA} e somente B utilizar \verb|~\ref{fig:labelFigB}|, ex. Figura~\ref{fig:labelFigB}.

\begin{figure}[!htp]
        \centering
        \begin{subfigure}[t]{0.4\textwidth}
                \includegraphics[width=\textwidth]{example-image-a}
                \caption{} % se quiser adicionar descrição ao lado de (a)
                \label{fig:labelFigA}
        \end{subfigure}       
        \begin{subfigure}[t]{0.4\textwidth}
                \includegraphics[width=\textwidth]{example-image-b}
                \caption{}
                \label{fig:labelFigB}
        \end{subfigure}
        %\vspace{-2\baselineskip} % ajustar a posição da legenda, se for necessário. Negativo aproxima da imagem, Positivo afasta
        \caption{Duas imagens e somente uma legenda}\label{fig:2imagens}
\end{figure}


\begin{figure}
\centering
  \begin{subfigure}[t]{.4\textwidth}
    \centering
    \includegraphics[width=\textwidth]{example-image-a}
    \caption{\textbf{Schnitt}: $A \cup B$: Element liegt in $A$ \textbf{oder} in $B$.}
  \end{subfigure}
  %\hfill % adicionar espaço entre as imagens
  \begin{subfigure}[t]{.4\textwidth}
    \centering
    \includegraphics[width=\textwidth]{example-image-b}
    \caption{\textbf{Vereinigung}: $A \cap B$: Element liegt in $A$ \textbf{und} in $B$.}
  \end{subfigure}
  \begin{subfigure}[t]{.4\textwidth}
    \centering
    \includegraphics[width=\textwidth]{example-image-c}
    \caption{\textbf{Differenz}: $A \setminus B$: Element liegt in $A$ \textbf{nicht} in $B$. (\textit{A ohne B})}
  \end{subfigure}
  %\hfill % adicionar espaço entre as imagens
  \begin{subfigure}[t]{.4\textwidth}
    \centering
    \includegraphics[width=\textwidth]{example-image-a}
    \caption{\textbf{Symmetrische Differenz}: $A \Delta B$: Element liegt \textbf{entweder} in $A$ oder in $B$.}
  \end{subfigure}
  \caption{Quatro imagens lado a lado}\label{fig:4imagens} 
\end{figure}



\begin{figure}
\centering
\begin{subfigure}[b]{.45\linewidth}
\includegraphics[width=\linewidth]{example-image-a}
\caption{A mouse}\label{fig:mouse}
\end{subfigure}
\begin{subfigure}[b]{.45\linewidth}
\includegraphics[width=\linewidth]{example-image-b}
\caption{A gull}\label{fig:gull}
\end{subfigure}
\begin{subfigure}[b]{.45\linewidth}
\includegraphics[width=\linewidth]{example-image-c}
\caption{A tiger}\label{fig:tiger}
\end{subfigure}
\caption{Picture of animals}
\label{fig:animals}
\end{figure}

%----------------------------------------------
\subsection{Indicar alterações no texto.}
%----------------------------------------------

Utilize \verb|\removido{}| para marcar um texto antigo, que será removido. Exemplo, \removido{texto a ser removido na próxima revisão.} 

Utilize \verb|\novo{}| para marcar os novos textos inseridos depois de uma revisão. Exemplo, \novo{Colocar aqui os novos textos inseridos, facilitando assim as revisões futuras.} 

Exemplo de Algoritmo~\ref{alg:alg1}.

\begin{algorithm} % enter the algorithm environment
\caption{Nome do algoritmo} % give the algorithm a caption
\label{alg:alg1} % and a label for \ref{} commands later in the document
\begin{algorithmic}[1] % enter the algorithmic environment
                       % [1] insere numeração de linhas
    \REQUIRE $n \geq 0 \vee x \neq 0$
    \ENSURE $y = x^n$ \COMMENT{some comment here}
    \STATE $y \Leftarrow 1$
    \IF{$n < 0$}
        \STATE $X \Leftarrow 1 / x$
        \STATE $N \Leftarrow -n$
    \ELSE
        \STATE $X \Leftarrow x$
        \STATE $N \Leftarrow n$
    \ENDIF
    \WHILE{$N \neq 0$}
        \IF{$N$ is even}
            \STATE $X \Leftarrow X \times X$
            \STATE $N \Leftarrow N / 2$
        \ELSE  [$N$ is odd]
            \STATE $y \Leftarrow y \times X$
            \STATE $N \Leftarrow N - 1$
        \ENDIF
    \ENDWHILE
    \FOR{xxxx}
        \STATE $X \Leftarrow x$
    \ENDFOR
\end{algorithmic}
\end{algorithm}



\begin{algorithm}
\caption{Calculate $y = x^n$}
\label{alg1}
\begin{multicols}{2}
\begin{algorithmic}[1]
  \REQUIRE $n \geq 0 \vee x \neq 0$
  \ENSURE $y = x^n$
  \STATE $y \Leftarrow 1$
  \IF{$n < 0$}
  \STATE $X \Leftarrow 1 / x$
  \STATE $N \Leftarrow -n$
  \ELSE
  \STATE $X \Leftarrow x$
  \STATE $N \Leftarrow n$
  \ENDIF
  \WHILE{$N \neq 0$}
  \IF{$N$ is even}
  \STATE $X \Leftarrow X \times X$
  \STATE $N \Leftarrow N / 2$
  \ELSE [$N$ is odd]
  \STATE $y \Leftarrow y \times X$
  \STATE $N \Leftarrow N - 1$
  \ENDIF
  \ENDWHILE
\end{algorithmic}
\end{multicols}
\end{algorithm}

