% Introdução
\chapter{Introdução}
\label{cap:capitulo1}
% label do capítulo para usar como referência no texto
% O Capítulo~\ref{cap:capitulo1} fala sobre

A introdução deve dar ao leitor o posicionamento da tese e a motivação suficiente para a leitura da tese, esclarecendo:
\begin{itemize}
	\item A natureza do problema cuja resolução se descreve (uma palhinha do problema);
	\item Uma indicação dos métodos usados para atacar o problema;
	\item As contribuições do trabalho e sua relevância para fazer progredir o estado da arte. Se for uma tese, aqui se coloca a hipótese de pesquisa, a tese, e como ela será demonstrada durante o texto;
	\item A forma como a tese está estruturada (estrutura do texto).
\end{itemize}

Uma dica para a montagem de uma tese ou monografia é responder as seguintes perguntas, cada uma em um parágrafo:
\begin{itemize}
	\item \textbf{O que fez?}: O que se propõe, o que se trabalhou, ``Propomos...'';
	\item \textbf{Como fez?}: Técnicas, metodologia de trabalho, ``Para tal...'';
	\item \textbf{Por que?}: Quais as motivações e justificativas do trabalho. A motivação inclui se faz parte de um projeto maior, o fato do problema não ter solução, e/ou qual a história que o levou a trabalhar nisso. A justificativa não é obrigatória, mas pode motivar o leitor;
	\item \textbf{Quanto vale?}: Quais as contribuições do trabalho. Inclui hipótese, tese,  e resultados principais (técnicas, métodos novos, produtos, publicações);
	\item \textbf{Para quem?} Quais as aplicações que se beneficiarão do seu trabalho, para quê será (ou poderá ser) usado - como parte de um projeto;
	\item \textbf{Como está?} Estrutura do texto. Evite escrever coisas obvias (como ``na Introdução, introduzimos...'', ``na conclusão, concluímos''). Coloque texto informativo, que acrescente o saber de quem lê (como ``na seção 2 será mostrado por que \textbf{wavelets} podem ser usadas para estabilizar corrente'').
\end{itemize}

Uma observação importante é não colocar ``Objetivos'' nos trabalhos finais, pois não são mais objetivos, uma vez que você os atingiu. Neste caso, estes viram contribuições. Os ``Objetivos'' são mais utilizados para qualificações e projetos de pesquisa, que ainda não foi desenvolvidos. Tomar cuidado também para colocar objetivos factíveis (porque será um problema se não os cumprir...).

\section{Problema de pesquisa}
\label{sec:problema}

\section{Motivação}


\section{Objetivos}

A pesquisa realizada é baseada nos objetivos geral e específicos que são apresentados a seguir.

\subsection{Objetivo Geral}

\subsection{Objetivos Específicos}

\begin{itemize}
    \item Objetivo 1;
    \item Objetivo 2;
    \item Objetivo 3;
    \item Objetivo 4;
    \item Objetivo 5;
\end{itemize}

\section{Solução Proposta}
\label{sec:solucao}

\section{Metodologia}
\label{sec:metodologia}

\section{Estrutura do trabalho}

Os capítulos desta dissertação estão organizados da seguinte forma:

\textbf{Capítulo~\ref{cap:capitulo1} -} Introduz o problema, apresenta a motivação e os objetivos do trabalho.

\textbf{Capítulo~\ref{cap:capitulo2} -} Apresenta o referencial teórico sobre as tecnologias relacionadas a solução proposta na pesquisa.

\textbf{Capítulo~\ref{cap:capitulo3} -} Apresenta trabalhos relacionados a pesquisa e problemas existentes.

\textbf{Capítulo~\ref{cap:capitulo4} -} Apresentação da arquitetura proposta, com detalhamento das camadas.

\textbf{Capítulo~\ref{cap:capitulo5} -} Demonstra os resultados obtidos após utilização do protótipo.

\textbf{Capítulo~\ref{cap:capitulo6} -} Considerações finais, breve apresentação dos resultados obtidos, limitações e dificuldades encontradas e trabalhos futuros. 

