% Capítulo 5
\chapter{Avaliação dos resultados ou \\ Resultados preliminares}
\label{cap:capitulo5}
% label do capítulo para usar como referência no texto
% O Capítulo~\ref{cap:capitulo5} fala sobre


\section{Seção A}

Seção A


\section{Seção A}

Alguns exemplos de citação: 

Na tese de Doutorado de Paquete \cite{PaquetePhD}, discute-se sobre algoritmos de busca local estocásticos aplicados a problemas de Otimização Combinatória considerando múltiplos objetivos. Por sua vez, o trabalho de \cite{KnowlesBoundedLebesgue}, publicado nos anais do IEEE CEC de 2003, mostra uma técnica de arquivamento também empregada no desenvolvimento de algoritmos evolucionários multi-objetivo, trabalho esse posteriormente estendido para um capítulo de livro dos mesmos autores \cite{KnowlesBoundedPareto}. Por fim, no relatório técnico de \citeonline{Jaszkiewicz}, fala-se sobre um algoritmo genético híbrido para problemas multi-critério, enquanto no artigo de jornal de Lopez \textit{et al.} \cite{LopezPaqueteStu} trata-se do \textit{trade-off} entre algoritmos genéticos e metodologias de busca local, também aplicados no contexto multi-critério e relacionado de alguma forma ao trabalho de Jaszkiewicz (\citeyear{Jaszkiewicz}).

Outros exemplos relacionados encontram-se em \cite{Silberschatz} (livro), \cite{DB2XML} (referência da Web) e \cite{Angelo} (dissertação de Mestrado).

\subsection{Subseção X}

Subseção X


\subsection{Subseção Y}

Subsection Y


\section{Seção C}

Seção C